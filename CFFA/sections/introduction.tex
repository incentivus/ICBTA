Time flew since Nakamato in the Bitcoin genesis block \cite{nakamoto2019bitcoin} mentioned the New York Times article about chancellor bailout \cite{times}. Again, in the third Bitcoin halving, the history has been repeated. The block number 630000 contains the message ``NYTimes 09/Apr/2020 With \$2.3T Injection, Fed's Plan Far Exceeds 2008 Rescue''. To crypto lovers it means the enunciation of decentralization. The path towards self-sovereignty, independence, and removing the governance authorities from critical tasks. Through out these years, decentralized finance (DeFi) attracts attention which unfortunately is not a practical alternative for traditional finance yet. 
This work, tries to fill this gap and bring the DeFi closer to the market demands.


Hashed time-lock contracts (HTLC) are the chief core of many new developments in the DeFi world. They proved their potential even before DeFi where Poon et. al introduced lightning network \cite{poon2016bitcoin}. 

Swaption is first introduced by Liu as the component made of two HTLCs to trade option in a peer-to-peer setting~\cite{liu2018atomic}. Later on, Tefagh~\etal~utilized swaptions to make the first non-custodial capital-free bond under first generation blockchains such as Bitcoin \cite{tefagh2020atomic}. In this sequel, we further analyse swaptions and futures markets in detail, and make several new useful components using HTLCs and the ideas behind swaptions architecture and put a step forth by theoretically proving their safety in the last section.


Until now, there have been so many trials to re-implement the features of fiat money market with blockchain backbones. Invention of Ethereum, proved the possibility of writing smart contracts \cite{buterin2014next}. Afterward, thousands of successful Dapp projects were implemented on top of Ethereum aiming the same goals as ours, including MakerDAO, a governance token and executive voting based loan system\cite{maker}, Compound, a collateralized asset-lending platform \cite{compound}, dYdX, a trading platform for crypto assets \cite{dydx}, Aave, a depositing and borrowing money market protocol \cite{aave} and Ethereum wrapped tokens such as wrapped bitcoin \cite{wbtc}, to name a few.


Furthermore, several different blockchains were implemented, such as Cardano, the first peer-reviewed proof of stake blockchain \cite{kiayias2017ouroboros}, Cosmos, a network of independent parallel blockchains \cite{kwon2018network}, Monero, a private and untraceable currency \cite{van2013cryptonote}, Zcash, a strongly private currency with low fees \cite{sasson2014zerocash}, Ripple, a peer-to-peer platform for transferring money \cite{schwartz2014ripple}, Stellar, whose consensus protocol depends on a set of selected honest nodes rather than all the nodes \cite{mazieres2015stellar}. This growing diversity aroused the need for interoperatable protocols. The idea of atomic swap using HTLCs as the first proposed protocol is first suggested by Nolan in bitcointalk \cite{htlc-btctalk}. Later, Herily formalized the atomic cross-chain swaps \cite{herlihy2018atomic}. There are analyzes on these kind of swaps like Xu \etal analyzed the success rate of HTLC-based cross-chain atomic swaps using a game-theoretic approach \cite{xu2020game}. Zie~\etal~extended the approach in a way that it only needs multi-signature support, so it can be implemented in blockchains lacking HTLCs. However, it needs smart contracts on the other chain as previous approaches did  \cite{10.1007/978-3-030-31500-9_14}.

Many works have been trying to either formally prove the security of atomic swaps, or to find an attack to compromise the safety of its participating parties. For example, Meyden used a multi-agent setting to assess functionality of atomic swap \cite{van2019specification}. Then, Hirai analysed the atomic swap protocol as an asynchronous communication setting using the Kripke modal logic \cite{10.1007/978-3-030-03427-6_29}. Han~\etal~also tried to provide fairness for atomic swaps, however their proposal needed smart contract \cite{han2019optionality}.
Tsabary~\etal~devised a new attack based on miners incentive \cite{Tsabary2020MADHTLCBH}. In an HTLC, one party can incentivise miners to mine the later transaction by suggesting larger fees. They also designed the MAD-HTLC to remove this vulnerability from atomic swap contracts.

Afterward, many high-level protocols implemented atomic swaps. Komodo community has implemented its own unique variation of atomic swaps, named AtomicDEX \cite{AtomicDEX}. Some other swap protocols are implemented like \eg~UniSwap \cite{adams2020uniswap}, BurgerSwap \cite{burger}, SushiSwap \cite{sushi} and KyberSwap \cite{kyberswap} which swap ERC20 tokens. Jellyswap, besides offering swaps on different chains, provides liquidity for its protocol by designing Butler \cite{butler}. This way, every market maker gets profit with no restriction. However, these food-themed swaps are mostly based on \emph{automated market maker} (AMM) and yield farming not on atomic swaps. In AMM, liquidity providers get shares in fees or amount of governance tokens in return of injecting money to liquidity pools. These pools generate a marketplace where users can lend, borrow, or exchange tokens. yield farmers make money by trying to choose the best pools to invest in. Atomex \cite{atomex} and liquality \cite{liq} also provide atomic swap service but support a rather limited number of coins. Black~\etal~ proposed a new loan platform on Ethereum which accepts collateral in Bitcoin network \cite{black2019atomic}. Duggirala~\etal~designed an atomic swap using zero knowledge prove, increasing the privacy and security of participating parties in contract \cite{zk-sw}.
These newly devised protocols emphasize on the ever increasing demand for cross-chain protocols. 

% Jelly also offers the opportunity to provide liquidity and earn profit. Market Makers can gain profit from price spreads with no restrictions and the ability to get out of the protocol at any time. This is possible due to the instant OTC (Over-The-Counter) market making software developed by our team and called Buttler.

The significance of this work can be categorized as follows:
\begin{enumerate}
    \item First of all, in section~\ref{sec:swaption} we formalized the notion of atomic swaptions component and made a general form of them called \emph{meta swaption} in which no smart contracts are needed and limited scrips like bitcoin script are sufficient. Then, using this meta form, we implemented several instances of swaptions depending of need, including late deposition and margin-free, despite the belief that there can not exist any margin-free swaption due to the limitation of HTLCs. 
    
    \item Employing the implemented swaptions, in section~\ref{sec:arbitrage} we designed arbitrage opportunities on swaption market. This was not possible without our new techniques \emph{option delegation} and \emph{overlapping}. Overlapping helps reduce the cost of arbitrages and option delegation solves the multi-leader problem first mentioned in \cite{herlihy2018atomic} and is independent of smart contracts.
    
    \item Finally, in section~\ref{sec:tangle} we introduce a network of swaptions and arbitrages that are in interconnection with each other. This network is called a tangled money market. Then, two theorems are proved to describe the operation of them and two protocols to solve the problem in either settings. The first theorem explains how several swaptions can be executed in concurrency and if a party participating in such a network follows our protocol, his safety is guaranteed. The second theorem, gives a general statement in all types of possible tangled money markets with any topology while proving the safety is provided for complying parties to our protocol. 
    
    
\end{enumerate}

% Things were going well during my second year at the university, then, in the blink of an eye, my world was turned upside-down.


