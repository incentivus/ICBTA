\section{Conclusion}

In this paper, we first introduced \newfateme{the needed requirements of an atomic bond service using the general overview of ABCD.} \newfateme{Afterwards,} we derived ABCD to achieve the goal of providing an interoperable cross-chain bond. \newfateme{Finally, by extending its design, we empowered the ABCD primitive to resist the market fluctuations. All of the different scenarios of taking part in an ABCD protocol is tested on the Bitcoin testnet. Implementation of ABCD and also a pointer to transactions spent, are available in} \new{\cite{abcd-ref}}.
Collectively, we have employed the well-known atomic cross-chain swaps for building ABCD as a primitive for uncollateralized DeFi. Potential use cases include but are not limited to exploiting arbitrage opportunities between swaptions without owning any capital or any other similar use case of flash loans and flash swaps with two main improvements: 
\begin{itemize}
    \item Despite the similarities, instead of being a ``flash'' loan which must get repaid within a block, \abcd can span an arbitrarily long period for the issuer to trade or invest with the capital before the bond reaches maturity. The significance of this feature \newfateme{unfolds} by noting that this is not possible even in conventional financial systems to have an unsecured debt without a credit system. More precisely, this is only possible due to the full transparency and traceability of cryptocurrencies.
    \item Our proposed bond primitive does not require a Turing-complete programming language. The Bitcoin scripting language is sufficient to implement our method, which \newfateme{only relies} on HTLC. While most DeFi protocols rely heavily on \newfateme{smart contract custody} or third parties that make them susceptible to security issues, \abcd can be flexibly used on the wide variety of HTLC-compatible blockchains and, in particular, supports Bitcoin and its lightning network natively.
\end{itemize}
