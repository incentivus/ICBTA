\section{Conslusion}

In this paper, we first introduced ABD which is a single-chain bond contract. Afterwards by extending its design, we derived ABCD to achieve the goal of providing an interoperable cross-chain bond. Collectively, we have employed the well-known atomic cross-chain swaps for building ABCD as a primitive for uncollateralized DeFi. Potential use cases include but are not limited to exploiting arbitrage opportunities between swaptions without owning any capital or any other similar use case of flash loans and flash swaps with two major improvements: 
\begin{itemize}
    \item Despite the similarities, instead of being a ``flash'' loan which must be repaid within a block, \abcd can span an arbitrary long period of time for the issuer to trade or invest with the capital before the bond reaches maturity. The significance of this feature is highlighted by noting that this is not possible even in conventional financial systems to have an unsecured debt without a credit system. More precisely, this property is only possible due to full transparency and traceability of cryptocurrencies.
    \item Our proposed bond primitive does not require a Turing-complete programming language and bitcoin scripting language is sufficient to implement our method which is completely based on HTLC. While most DeFi protocols rely heavily on smart contracts or third parties which make them susceptible to security issues, \abcd can be flexibly used on the wide variety of HTLC-compatible blockchains and in particular, supports bitcoin natively.
\end{itemize}
 